\documentclass[../main.tex]{subfiles}
\graphicspath{{\subfix{../images/}}}
\begin{document}
\section{Marche à suivre de l'expérience de JJ Thomson}

\subsection{But :}
Observer la présence d'électrons

\subsection{Hypothèses de départ :}
On suppose que 
\subsection{Marche à suivre :}

\begin{enumerate}
    \item Produire une forte tension électrique entre les deux électrodes qui se situent au bout du tube en verre qu'on aura préalablement vidé une partie de son air.
    \item Il y aura 
\end{enumerate}

\subsection{Résultats :}
Le faisceau lumineux qui est créé par l'écran fluorescent et le rayon cathodique a deux option : si il n'y a aucune influence extérieure (champ magnétique, électrique...) il ira tout droit, mais s'il y a un champ électrique (qui pourrait être causé par le champ magnétique d'un fort aimant), le faisceau va changé de trajectoire en direction de la plaque positive.\\
Thomson a également trouvé le rapport entre la masse et la charge de la particule.

\subsection{Discussion :}
Le faisceau ne peut être que composé de particules, cela ne peut pas être de la lumière vu qu'elle ne change pas de trajectoire en présence de champ électrique ou magnétique. Et vu que le faisceau lumineux dévie en direction de la plaque positive, il doit être chargé négativement vu que le plus attire le moins. Les particules négatives qui composent le faisceau doivent être à l'extérieur du noyau.
\subsection{Conclusion :}

Il existe des particules de charges négatives capables de quitter le noyau, on les appelle électrons. La vitesse des électrons dans son tube de Crookes est le dixième de celle de la lumière, vitesse qui n'a quasiment jamais été atteinte pour l'époque.   
\end{document}