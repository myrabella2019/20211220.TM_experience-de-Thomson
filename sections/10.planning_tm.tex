\documentclass[../main.tex]{subfiles}
\graphicspath{{\subfix{../images/}}}
\begin{document}
\section{Planning TM au jour le jour
}
    
\begin{tabular}{|p{1.5cm}|p{1cm}|p{0.7cm}|p{1cm}|p{4cm}|p{1cm}|c|c|p{2cm}|}
    \hline
    Date (fait quand ?) & Heure & Ordre & Matière & Sujet & Date (prévu pour) & Début & Fin & Commentaires\\
    \hline
    20.12.21 & 20h à 23h & 1 & TM & création du document Latex et de toutes les fiches &  &  &  &  \\
    \hline
    21.12.21 & 18h à 23h & 2  & TM & modification du plan de TM (esquisse) et du mémento &  &  &  & \\
    \hline
    23.12.21 & 22h à 0h & 3 & TM & tentative de marche à suivre de l'expérience, écriture de la source du livre de chimie non reformulée  &  &  &  &  \\
    \hline
    24.12.21 & 18h à 22h & 4 & TM & retentative de créer une marche à suivre &  &  &  &  \\
    \hline
    25.12.21 & 11h à 18h & 5 & TM & écriture du paragraphe 16.10 sur l'oscilloscope du livre de physique non reformulé, du 19.9 &  &  &  &  \\
    \hline
    26.12.21 & 10h à 21h & 6 & TM & écriture du paragraphe 16.10 sur l'oscilloscope et 16.7 sur la capacité non reformulé &  &  &  &  \\
    \hline
    27.12.21 & 12h à 22h & 7 & TM & écriture du paragraphe 16.4 sur le potentiel électrique, métaphore avec l'escalateur. vidéo de van biezen sur thomson (prise de note dans le cahier), écriture de connaissances requises &  &  &  &  \\
    \hline
    28.12.21 & 11h à 21h & 8 & TM & écriture du paragraphe 16.10 sur l'oscilloscope  &  &  &  &  \\
    \hline
    29.12.21 & 16h à 01h & 9 & TM & écriture de 14.avec_bouteille (maquette de l'expérience) en utilisant des images pour décrire &  &  &  &  \\
    \hline
    30.12.21 & 9h45 à h & 10 & TM & document 16. rapport de l'expérience &  &  &  &  \\
    \hline
\end{tabular}


\end{document}