\documentclass[../main.tex]{subfiles}
\graphicspath{{\subfix{../images/}}}
\begin{document}
\subsection{Esquisse du TM}

\subsection{Sujet}
%Définir le sujet et donner un titre provisoire
L'Expérience de Thomson réalisée avec une démarche différente 

\subsection{Objectifs}

%Justification et motivation des choix

Je choisis de faire l'expérience de Thomson, car c'est une expérience assez complexe et qui est primitive à la chimie classique, en effet c'est la découverte des électrons. Cette expérience mélange à la fois la physique et la chimie. Elle reprend le thème de l'électricité en physique et en chimie elle reprend la composition de l'atome.
 
 
%\subsection{Contexte historique}
%\subsubsection{c'est quoi le problème ?}
%le problème était qu'on ne savait pas où partait les charges électriques. On ne connaissait que l'existence des protons, on ne connaissait pas encore les neutrons. 

%\subsubsection{la science en était où ?}
%Avant même l'an 0, on commencait déjà à s'intéresser à la matière.
%Les philosophes faisaient déjà des hypothèses sur comment pourrait être former la matière, est-ce qu'elle a plusieurs couches etc. \\
%Démocrite par exemple avait déjà commencé à donner des noms aux "petites particules". Plein d'autres philosophes ont suivi ce mouvement.\\
%Puis il y a eu la découverte des premieres éléments par Mendeleïv qui a ensuiteélaboré la sutrucre du tableau périodique que l'on connait aujourd'hui.

%Puis il y a eu une véritable course à la recherche de nouveaux éléments pour compléter le tableau. Des chimistes (et alchmistes) se sont plus penchés sur ce tableau, en essayant d'analyser pourquoi une tel matière étati à cette place, c'est alors qu'ils se sont penchés sur l'intérieur de l'atome en lui-même.

%On connaissait déjà l'existence des protons. Les neutrons seront découverts par Rutherford après Thomson pour les électrons. 

%\subsubsection{Les blocages de la science }
%(est-ce que la science -les scientifiques- savait, était-elle consciente qu'elle était bloquée)\\

%La science était bloquée au niveau de comment les charges électriques disparaissaient, on ne connaissait que l'existence de charges électriques positives, mais on ne comprenait pas l'interaction entre les atomes, pourquoi parfois ils se repoussaient ou se rapprochaient d'autres fois ?

%\subsubsection{Thomson avant son expérience ; qu'est-ce qu'il en pense de tout ça}
%aucune idée, à rechercher 
 
\subsection{Problématique}

%Définir la question générale; définir les différentes orientations possibles ; choisir des mots-clés permettant d'orienter la recherche.
\textbf{Question générale :} \\
Reconstituer l'expérience de Thomson en ayant les mêmes buts que J.J Thomson, c'est-à-dire de savoir si dans le tube de Crooks, il y avait des ondes ou un faisceau de particules dont on ne connaît pas la charge.\\
\textbf{Orientation :}\\
La démarche de l'expérience s'oriente vers l'électricité et vers le magnétisme\\
\textbf{Mots-clés :}
 tube de Crooks,
 Éléctricité, polarité, 
 Force magnétique.

\subsection{Bibliographie }

%Donner une liste provisoire des références (livres, sites Internet, films, ...) concernant le sujet choisi
\textbf{Livres :}\\
\begin{enumerate}
    \item KANE Joseph, STERNHEIM Morton, Physique, édition Dunod, Licence - Paes
    \item Physique 1, généralités chaleur, édition Lep
    \item BENSON Harris, \textit{Physique, 2. Électricité et magnétisme}, 4e édition, Canada, de Boeck, 1991, p.1-59 (force et champ électriques) p.109-140 (potentiel électrique) p.269-302 (champ magnétique)
    \item LAFRANCE René,\textit{Physique 2, Éléctricité et magnétisme}, Montréal (Québec), de Boeck, 2014, p.2-71 (charge et champ électriques) p.118-154 (potentiel électrique) p.252-328 (force et champ magnétique)
    \item REBSTEIN Martine et SOERENSEN Chantal, \textit{Chimie - préparation au bac et à la maturité}, 2e édition revue et augmentée, Suisse, Presses polytechniques et universitaires romandes, 2008
\end{enumerate}
\textbf{Sites Internet :}\\
\begin{enumerate}
    \item \href{https://fr.khanacademy.org/science/chemistry/electronic-structure-of-atoms/history-of-atomic-structure/a/discovery-of-the-electron-and-nucleus}{Khan academy, consulté la dernière fois 16.01.2022}
    \item \href{https://chemistrygod.com/cathode-ray-tube-experiments}{ChemistryGod - tube cathodique, consulté la dernière fois 16.01.2022}\\
\end{enumerate}
\textbf{Ressources vidéos :}\\
\begin{enumerate}
    \item \href{https://www.youtube.com/watch?v=WHoH5m83Ga0}{Académie de Bordeaux, \textit{[Histoire des sciences] La découverte de l'électron}, scientifique, France, CEA Recherche, 15.09.2014}
    \item \href{https://www.youtube.com/watch?v=Ka3v5dIQGOI}{-,-, \textit{Crooke's Tube \& Electrons}, LHSAtkins}
    \item \href{https://www.youtube.com/watch?v=GR9A7Hd4mxQ}{-, -, \textit{JJ Thomson and the discovery of the electron}, PhysicsHigh}
    \begin{enumerate}[1. ] 
    \item \href{h https://www.youtube.com/watch?v=-Rb9guSEeVE&list=PLkyBCj4JhHt8DFH9QysGWm4h_DOxT93fb}{Electric Potential: Visualizing Voltage with 3D animations}
    \item \href{https://www.youtube.com/watch?v=f_MZNsEqyQw&list=PLkyBCj4JhHt8DFH9QysGWm4h_DOxT93fb&index=16}{Capacitors and Capacitance: Capacitor physics and circuit operation}
     \item \href{https://www.youtube.com/watch?v=P0Jnx1BjIZM&list=PLkyBCj4JhHt8DFH9QysGWm4h_DOxT93fb&index=29}{Electric Flux Paradox}
     \item \href{https://www.youtube.com/watch?v=eMTuPhg-8Go&t=22s}{Courant électrique : tension et intensité}
     \item \href{https://www.youtube.com/watch?v=m4jzgqZu-4s}{Electric Circuits: Basics of the voltage and current laws.}
     \item \href{https://www.youtube.com/watch?v=u4FpbaMW5sk}{Battery Energy and Power}
      \item \href{https://www.youtube.com/watch?v=5YYVnHN7xwM}{Charge to mass ratio of an electron}
      \item \href{https://www.youtube.com/watch?v=Ka3v5dIQGOI}{Crooke's Tube & Electrons}
       \item \href{https://fr.wikipedia.org/wiki/Bobine_Tesla}{Tesla Bobine COIL}
      \item \href{https://www.youtube.com/watch?v=QuLtuM8bAMI}{Demonstrating JJ Thomson experiment on charge to mass ratio}
      \item \href{https://www.youtube.com/watch?v=GR9A7Hd4mxQ}{JJ Thomson and the discovery of the electron}
      \item \href{https://www.youtube.com/watch?v=i6zyPOSreCg}{Cathode Ray Tube Experiment and Charge To Mass Ratio of an Electron}
    \item \href{https://www.youtube.com/watch?v=Rb6MguN0Uj4}{Discovery of the Electron: Cathode Ray Tube Experiment}

    \item \href{https://www.webassign.net/question_assets/unccolphyseml1/lab_4/manual.html}{Determination of e/m for the Electron}
    \item \href{https://physicsx.erau.edu/HelmholtzCoils/Lab_MP_1.pdf}{e/m ratio of the electron}
    \item \href{https://virtuelle-experimente.de/en/b-feld/e-m-bestimmung/edurchm.php}{Use the experiment to measure the ratio of the electrons charge to it's mass.}{}
    \item \href{https://www.physik.uzh.ch/~matthias/espace-assistant/manuals/en/anleitung_etom_e.pdf}{4. Electron Charge-to-Mass Ratio e/m - UZH - Physik-Institut}
    \item \href{https://demoweb.physics.ucla.edu/6b-lab-manual}{UCLA Physics \& Astronomy}
     \item \href{https://www.famousscientists.org/j-j-thomson/}{J. J. Thomson. Famous Scientists -The Art of Genius-}
     \item \href{https://virtuelle-experimente.de/en/index.php}{Electron Motion in Electric and Magnetic Fields}
\end{enumerate}

\textbf{Van Biezen - Magnetic fields and magnetic}
\begin{enumerate}[A.]
    \item \href{https://www.youtube.com/watch?v=kBoasyx8C_Y&list=PLX2gX-ftPVXX3FUB8FPKFPPXPJ6yhY4mT}{Physics - Magnetic Forces on Moving Charges - Direction (1 of 6) An Introduction}
    \item \href{https://www.youtube.com/watch?v=FXVJHjrgDuE&list=PLX2gX-ftPVXX3FUB8FPKFPPXPJ6yhY4mT&index=2}{Physics - Magnetic Forces on Moving Charges - Direction + Magnitude (2 of 6)}
    \item \href{https://www.youtube.com/watch?v=c6bcGfdLZ7c&list=PLX2gX-ftPVXX3FUB8FPKFPPXPJ6yhY4mT&index=3}{Physics - Magnetic Forces on Moving Charges - Direction + Magnitude (3 of 6)}
    \item \href{https://www.youtube.com/watch?v=DeAU6IQH0R4&list=PLX2gX-ftPVXX3FUB8FPKFPPXPJ6yhY4mT&index=4}{Physics - Magnetic Forces on Moving Charges - Direction + Magnitude (4 of 6)}
    \item \href{https://www.youtube.com/watch?v=qmzdN55zpTE&list=PLX2gX-ftPVXX3FUB8FPKFPPXPJ6yhY4mT&index=5}{Physics - Magnetic Forces on Moving Charges - Direction + Magnitude (5 of 6)}
    \item \href{https://www.youtube.com/watch?v=pw8seJUQ6VA&list=PLX2gX-ftPVXX3FUB8FPKFPPXPJ6yhY4mT&index=6}{Physics - Magnetic Forces on Moving Charges - Direction + Magnitude (6 of 6)}
    \item \href{https://www.youtube.com/watch?v=rCc5-IxUXEI&list=PLX2gX-ftPVXX3FUB8FPKFPPXPJ6yhY4mT&index=7}{Physics - E&M: Magn Field Effects on Moving Charge & Currents (7 of 26) F=? Current Loop}
    \item \href{https://www.youtube.com/watch?v=bc2sjL1wVDo&list=PLX2gX-ftPVXX3FUB8FPKFPPXPJ6yhY4mT&index=8}{Physics - E&M: Magn Field Effects on Moving Charge & Currents (8 of 26) Torque=? Current Loop}
    \item \href{https://www.youtube.com/watch?v=um_gDo3lUHs&list=PLX2gX-ftPVXX3FUB8FPKFPPXPJ6yhY4mT&index=9}{Physics - E&M: Magn Field Effects on Moving Charge & Currents (9 of 26) How Torque Changes}
    \item \href{https://www.youtube.com/watch?v=2Zi7ekMAZYU&list=PLX2gX-ftPVXX3FUB8FPKFPPXPJ6yhY4mT&index=10}{Physics - E&M: Magn Field Effects on Moving Charge & Currents (10 of 26) Magnetic Dipole Moment}
    \item \href{https://www.youtube.com/watch?v=urEBAiTr_-k&list=PLX2gX-ftPVXX3FUB8FPKFPPXPJ6yhY4mT&index=11}{Physics - E&M: Magn Field Effects on Moving Charge & Currents (11 of 26) P.E. of Magnetic Dipole}
    \item \href{https://www.youtube.com/watch?v=aqOY66dw1qU&list=PLX2gX-ftPVXX3FUB8FPKFPPXPJ6yhY4mT&index=12}{Physics - E&M: Magn Field Effects on Moving Charge & Currents (12 of 26) P.E.=? of Magnetic Dipole}
    \item \href{https://www.youtube.com/watch?v=fWaxrZHK5BY&list=PLX2gX-ftPVXX3FUB8FPKFPPXPJ6yhY4mT&index=13}{Physics - E&M: Magn Field Effects on Moving Charge & Currents (13 of 26) Force on a Current}
    \item \href{https://www.youtube.com/watch?v=UK4eBzmfsY0&list=PLX2gX-ftPVXX3FUB8FPKFPPXPJ6yhY4mT&index=14}{Physics - E&M: Magn Field Effects on Moving Charge & Currents (14 of 26) Force=? on a Current}
    \item \href{https://www.youtube.com/watch?v=YtN7_A8MeyE&list=PLX2gX-ftPVXX3FUB8FPKFPPXPJ6yhY4mT&index=15}{Physics - E&M: Magn Field Effects on Moving Charge & Currents (15 of 26) Torque=? Circular Current}
    \item \href{https://www.youtube.com/watch?v=DnHVgItjzu8&list=PLX2gX-ftPVXX3FUB8FPKFPPXPJ6yhY4mT&index=16}{Physics - E&M: Magn Field Effects on Moving Charge & Currents (16 of 26) The Velocity Selector}
    \item \href{https://www.youtube.com/watch?v=3fUjO2BoNFo&list=PLX2gX-ftPVXX3FUB8FPKFPPXPJ6yhY4mT&index=17}{Physics - E&M: Magn Field Effects on Moving Charge & Currents (17 of 26) Thompson's e/m Ratio}
    \item \href{https://www.youtube.com/watch?v=AR654M0Ruro}{Understanding Charge in magnetic field}
    \item \href{https://www.youtube.com/watch?v=fPST273JymU}{understanding charge in an electric field}
    \item \href{https://www.youtube.com/watch?v=vXOeehVTcRA}{Cathode Ray Tube | www.MyInterAcademy.com}
    \item \href{https://www.youtube.com/watch?v=8PRF-hrbAjk}{Class 9 Science - Structure of the Atom | Charged Particles in Matter}
     \item \href{https://www.youtube.com/watch?v=gTRVtDXMs4s}{From Geissler Tubes to Cathode Ray Tubes (Crookes Tubes), Physics & History}
\end{enumerate}
\textbf{Van Biezen - Physics 36 The Electric Fields}
\begin{enumerate}
    \item \href{https://www.youtube.com/watch?v=EPIhhbwbCNc&list=PLX2gX-ftPVXUcMGbk1A7UbNtgadPsK5BD&index=9}{Physics 36 The Electric Field (1 of 18)}
\end{enumerate}

\end{enumerate}
\subsection{Plan de travail}

%effectuer un échéancier des différentes étapes de l'élaboration du travail de maturité

\subsection{Introduction :}
Le sujet choisi est intéressant car tout d'abord, l'expérience de Thomson est à la base de la chimique classique comme le modèle de Dalton, l'expérience de Rutherford ou encore celle de Milikan. En effet, cette expérience nous a permis de découvrir la présence des électrons dans les atomes, elle est donc légitime même pour la physique classique, car elle utilise des notions d'électricité et de magnétisme. Du point de vue de l'histoire, on se situe à la fin du XIXe - début du XXe siècle, c'est donc l'industrialisation. Scientifiquement, on sait que la matière est composée d'atomes, Mendeleïev a crée son tableau, c'est une période de "chasse aux éléments". Dalton a proposé son modèle, mais les anglais et les allemands ne sont pas d'accord sur ce qu'il se passe, on ne savait pas de quelle nature était le faisceau, ondes ou particules ? Ce sujet est motivant car c'est toujours curieux de savoir comment ont été découvert les éléments fondamentaux, de plus l'expérience a un beau résultat, on peut "jouer" en quelque sorte avec la direction du faisceau cathodique en modifiant les paramètres. La problématique est donc de savoir la nature du faisceau cathodique et d'en étudier la composition, le but est aussi de trouver le rapport de la masse et de la charge d'un électron et d'en déterminer la vitesse. La méthode pour trouver cela est tout d'abord de réaliser l'expérience et d'en prendre des mesures. Grâce aux lois reliées à l'électricité et au magnétisme. Les limites du sujet sont le modèle de Dalton et le tube de Crooks qui est primitif à cette expérience. L'expérience est réalisable dans une salle de TP, il faudra se procurer le matériel nécessaire.
\subsection{Exécution :}
\textbf{Partie théorique :} Pour atteindre le but, il faut expliquer les différents concepts de l'électricité (force, champ, potentiel voir plus) et le magnétisme (force, champ voir plus). Ces notions seront expliquées dans un cas général.\\
\textbf{Partie pratique :} expérience réalisée en salle de TP\\
En annexes seront présent des schémas de l'expérience, des schémas des différents exemples exploitées dans la partie théorique qui permettront de mieux expliquer et de clarifier les propos. 
\subsection{Résultats :}
Les résultats seront les mesures apportés par l'expérience. Ces dernières seront prises sur l'écran fluorescent. Comme graphiques, on pourrait comparer la position du faisceau sur l'écran selon l'intensité du champ des plaques métalliques, selon l'intensité du champ des aimants.
\subsection{Analyse}
Dans cette partie, on fera les calculs pour trouver le rapport de la masse et de la charge. Si l'expérience n'aurait pas fonctionné, pourquoi ?\\
\vspace{2cm}
Un glossaire est pertinent pour expliquer les termes scientifiques.
\end{document}