\documentclass[../main.tex]{subfiles}
\graphicspath{{\subfix{../images/}}}
\begin{document}
\subsection{Plan du TM}

\subsection{Sujet}
%Définir le sujet et donner un titre provisoire
L'Expérience de Thomson réalisée avec une démarche différente 

\subsection{Objectifs}

%Justification et motivation des choix

Je choisis de faire l'expérience de Thomson, car c'est une expérience très complexe et qui est primitive à la chimie classique, en effet c'est la découverte des électrons. C'est la base de tous les thèmes, comme par exemple les cations et les anions, les charges électriques. Ce dernier thème est même très abordé en physique classique. Il met en jeu .
 
 
\subsection{Contexte historique}
\subsubsection{c'est quoi le problème ?}
le problème était qu'on ne savait pas où partait les charges électriques. On ne connaissait que l'existence des protons, on ne connaissait pas encore les neutrons. 

\subsubsection{la science en était où ?}
Avant même l'an 0, on commencait déjà à s'intéresser à la matière.
Les philosophes faisaient déjà des hypothèses sur comment pourrait être former la matière, est-ce qu'elle a plusieurs couches etc. \\
Démocrite par exemple avait déjà commencé à donner des noms aux "petites particules". Plein d'autres philosophes ont suivi ce mouvement.\\
Puis il y a eu la découverte des premieres éléments par Mendeleïv qui a ensuiteélaboré la sutrucre du tableau périodique que l'on connait aujourd'hui.

Puis il y a eu une véritable course à la recherche de nouveaux éléments pour compléter le tableau. Des chimistes (et alchmistes) se sont plus penchés sur ce tableau, en essayant d'analyser pourquoi une tel matière étati à cette place, c'est alors qu'ils se sont penchés sur l'intérieur de l'atome en lui-même.

On connaissait déjà l'existence des protons. Les neutrons seront découverts par Rutherford après Thomson pour les électrons. 

\subsubsection{Les blocages de la science }
(est-ce que la science -les scientifiques- savait, était-elle consciente qu'elle était bloquée)\\

La science était bloquée au niveau de comment les charges électriques disparaissaient, on ne connaissait que l'existence de charges électriques positives, mais on ne comprenait pas l'interaction entre les atomes, pourquoi parfois ils se repoussaient ou se rapprochaient d'autres fois ?

\subsubsection{Thomson avant son expérience ; qu'est-ce qu'il en pense de tout ça}
aucune idée, à rechercher 
 
\subsection{Problématique}

%Définir la question générale; définir les différentes orientations possibles ; choisir des mots-clés permettant d'orienter la recherche.




\subsection{Bibliographie}

%Donner une liste provisoire des références (livres, sites Internet, films, ...) concernant le sujet choisi

\subsection{Plan de travail}

%effectuer un échéancier des différentes étapes de l'élaboration du travail de maturité

\end{document}