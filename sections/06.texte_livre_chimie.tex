\documentclass[../main.tex]{subfiles}
\graphicspath{{\subfix{../images/}}}
\begin{document}
\section{p.29 - Texte du livre sur l'expérience de Thomson}

En 1897, le physicien anglais Joseph J. Thomson (1856-1940) postula que l'électricité était formée de particules de matière beaucoup plus légère que l'atome le plus léger et qu'elles portaient une charge électrique négative. C'est en utilisant un tube cathodique de Crookes (fig 2.4) et à partir de l'expérience suivante qu'il fit ces déductions.\\
\subsection{Principe de fonctionnement}

Soit un tube en verre contenant :
\begin{itemize}
    \item deux électrodes (2 plaques métalliques), l'une négative et l'autre positive percée en son centre.
    \item un écran fluorescent
    \item 2 plaques métalliques pouvant produire un champ électrique
\end{itemize}
Si l'on applique une forte tension électrique entre les deux électrodes situées aux extrémités du tube en verre (partiellement vidé de son air (ou d'un autre gaz)), on observe une décharge accompagnée d'une émission de lumière. Sous cette tension, l'électrode négative émet un rayon (appelé \textbf{rayon cathodique}) qui est attiré par l'électrode positive et qui la traverse en son centre. Le rayon poursuit ensuite son trajet jusqu'à l'écran qzu se trouve à l'autre extrémité du tube. Au moment de l'impact avec l'écran, il se produit une forte fluorescence et une lumière vive apparaît. En l'absence d'influence extérieure, le faisceau lumineux se dirige en ligne droite (faisceau A). A ce stade de l'observation, le faisceau pourrait être causé soit par des rayons lumineux, soit par des particules qui partent de l'électrode négative, se dirigent vers l'électrode positive et traversent le trou de cette dernière. En présence d'un champ électrique, provoqué par exemple par le champ magnétique d'un fort aimant permanent, on observe que le  faisceau lumineux est dévié en direction de la plaque positive (faisceau B). Comme la lumière ne peut pas être déviée par un champ électrique, il ne peut s'agir que de particules.\\
Ces observations permettent donc à JJ Thomson de postuler que le rayon cathodique doit être formé de très petites particules chargées négativement (les électrons) qui sont émises par le métal de l'électrode négative lorsque celle-ci est soumise à une forte tension.\\
En mesurant la déviation provoquée par le champ électrique (ou le champ magnétique), JJ Thomson peut déterminer le rapport entre la masse et la charge de l'électron.\\
Il constate également que les propriétés des rayons sont les mêmes quel que soit le métal choisi pour la cathode. Il en déduit que les électrons sont des constituants fondamentaux de tous les atomes et que ces particules était attirées vers l'électrode positive, elles doivent être chargées négativement.
\subsection{Conclusion}
Des particules de charge négative peuvent quitter l'atome. On les nomme \textbf{électrons}.
\end{document}