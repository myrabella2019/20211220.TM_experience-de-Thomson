\documentclass[../main.tex]{subfiles}
\graphicspath{{\subfix{../images/}}}
\begin{document}
\section{Livre de chimie - préparation à la maturité}
p.29\\
\subsection{L'expérience de Thomson}
J.J. Thomson a dit en 1897 que l'éléctricité est formé de petites particules chargées négativement. Il a expérimenté son hypothèse et a trouvé des particules qui se détachaient des atomes, ça ne pouvais pas être les protons, il en a donc conclu que c'était des particules dont on ignorait encore l'existence : les électrons.\\
\textbf{Matériel pour aboutir à ce résultat :} \\
un tube en verre contenant : \\
\begin{enumerate}
    \item 2 électrodes (2 plaques métalliques), 1 négative et 1 positive percée en son centre
    \item 1 écran fluorescent
    \item 2 plaques métalliques pouvant produire un champ électrique
\end{enumerate}
\textbf{Déroulement de l'expérience :}\\
On applique une tension életcrique entre les 2 plaques (électrodes) qui sont aux extrêmes du tube en verre, préalablement mis partiellement vide. On va ensuite observer le rayon cathodique . Un tube cathodique est un tube de verre étanche presque entièrement sous vide. En appliquant une haute tension entre deux électrodes à une extrémité du tube, on crée un faisceau de particules qui part de la cathode (l'électrode chargée négativement) vers l'anode (l'électrode chargée positivement). Ces tubes sont appelés cathodiques parce que le faisceau de particules, ou "faisceau cathodique", part de la cathode. Ce faisceau est détecté grâce à une couche de phosphore qui recouvre la partie du tube qui est à l'opposé des électrodes. Le phosphore émet de la lumière quand il est frappé par le flux cathodique. 

\textbf{après... : }\\
Bien qu'initialement controversées, les découvertes de Thomson furent progressivement acceptées par la communauté scientifique. Les particules composant le faisceau cathodique ont reçu le nom d'électrons. La découverte des électrons réfuta la partie de la théorie atomique de Dalton qui supposait que les atomes sont des entités indivisibles. Il fallait à présent penser à un tout nouveau modèle atomique afin d'inclure l'existence des électrons. 
\end{document}