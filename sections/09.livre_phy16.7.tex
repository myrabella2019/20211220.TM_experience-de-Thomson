\documentclass[../main.tex]{subfiles}
\graphicspath{{\subfix{../images/}}}
\begin{document}
\section{16.7 - Capacité p.418}

Un des problèmes en électrostatique est de pouvoir emmagasiner des charges de manière à pouvoir les utiliser par la suite.\\
\par Supposons que deux conducteurs soient initialement neutres et que nous enlevions ensuite de faibles quantités de charge de l'un et les placions sur l'autre. Au cours de ces processus, il se développe une différence de potentiel. La quantité de charge (Q) qu'un corps peut emmagasiner pour une différence de potentiel donnée (V) dépend de ses caractéristiques physiques et est définie comme étant la capacité (C) du corps.\\
\par Plus la charge est grande et plus la capacité est grande, plus la différence de potentiel est faible et plus la capacité sera grande.\\
\par La capacité est une mesure de la charge transférée sous une différence de potentiel donnée. Autrement dit, la capacité électrique \textit{C} est proportionnelle à \textit{Q} et inversement proportionnelle à \textit{V}\\
$$C = \frac{Q}{V}$$\\
On trouve habituellement que le rapport est un constante indépendante de la charge.\\
\par Le transfert de charge entre deux conducteurs séparés par le vide ou pas un isolant permet d'accroître la capacité du système et est appelé un condensateur.\\
\par L'énergie requise pour séparer les charges est emmagasinée dans le condensateur. La capacité est donc également une mesure du pouvoir d'accumuler l'énergie.\\
\par On peut trouver des exemples de condensateurs dans la nature. Par exemple, les membranes des cellules séparent des fines couches d'ions dans le fluide à l'intérieur et à l'extérieur de la cellule. On peut, par conséquent, considérer que la membrane et les fluides adjacents possèdent une capacité électrique.\\
\par Les condensateurs sont largement utilisés dans les circuits électriques. On les emploie dans les circuits d'accord de radio et de télévision, dans les systèmes d'allumage de voitures et dans les circuits de démarrage de moteurs électriques. Les condensateurs influencent aussi la façon dont les courants changent au cours du temps. Dans les cellules nerveuses, le taux de transmission d'une impulsion nerveuse dépend de la capacité de la membrane. Nous discuterons certaines de ces applications dans des chapitres ultérieurs.\\
\par Pour un condensateur, les deux conducteurs ont des charges égales et opposées $+- Q$ et une différence de potentiel correspondante \textit{V}. Le rapport est la capacité électrique \\
$$C = \frac{Q}{V}$$\\
Remarquer que l'on définit \textit{V} comme étant le potentiel de la lame positive moins celui de la lame négative, de sorte que \textit{C} est positif.\\
\par L'unité de capacité est le \textit{farad} (F) : 1 F = 1 C $V^{-1}$. Puisque le coulomb est une unité élevée, le farad est grand lui aussi et la plupart des condensateurs ont de faibles valeurs si elles sont exprimées en farads. D'où l'intérêt de définir le microfarad et le picofarad, qui sont \\
$$1 \mu F = 10^{-6}F, \qquad 1pF = 10^{-12}F$$\\
La capacité d'un être humain isolé du sol est de l'ordre de grande d'une centaine de pF, mais celle-ci dépend d'une façon importante de l'isolation vis-à-vis de la terre.\\
\subsection{Le condensateur à lames parallèles}
Le condensateur le plus simple est composé de deux lames parallèles dans le vide. Les lames ont des surfaces \textit{A}, des charges $+-Q$ et sont distantes de \textit{l} (figure 16.22). D'après l'équation (16.10),\\
$$V = \frac{Ql}{\epsilon_0A}$$\\
Donc, la capacité $C = Q/V$ vaut \\
$$C = \frac{\epsilon_0A}{l}$$\\
Remarquer que \textit{C} dépend seulement de la géométrie du système et pas de la charge \textit{Q}. La capacité augmente avec la surface des lames et décroît avec leur éloignement. Ceci est également vrai pur des condensateurs de forme plus compliquée. Des exemples numériques de condensateurs à lames parallèles sont donnés au paragraphe suivant.\\
\subsubsection*{Figure 16.22}
Un condensateur à lames parallèles.
\end{document}