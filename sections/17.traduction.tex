\documentclass[../main.tex]{subfiles}
\graphicspath{{\subfix{../images/}}}
\begin{document}
\section{Traduction}

\href{https://physicsx.erau.edu/HelmholtzCoils/Lab_MP_1.pdf}{e/m ratio of the electron}
    
\subsection{Objectif pédagogique}
Étudier le rapport charge-masse (e/m) des rayons cathodiques (électrons) dans un tube cathodique (CRT). 
\subsection{Objectif expérimental}
Mesurer la déviation d'un faisceau d'électrons dans un champ électrique et magnétique croisé et déterminer le rapport charge/masse.  
\subsection{Appareil}
Un tube cathodique Tel-Atomic contenant :
\begin{enumerate}
    \item un "canon à électrons", 
    \item une grille pour mesurer la déviation du faisceau d'électrons, et 
    \item une grille de champ électrique pour dévier les électrons. 
    \item Une paire de bobines de Helmholtz pour produire un champ magnétique homogène.   
\end{enumerate}
Des voltmètres et des ampèremètres pour : \\
\begin{enumerate}
    \item surveiller les potentiels sur le canon à électrons et la grille de déviation, et
    \item surveiller le courant dans le champ magnétique (bobines de Helmholtz).
\end{enumerate}
\subsection{Méthode}
Premièrement, accélérer les électrons à travers le "canon à électrons".  Deuxièmement, les dévier à l'aide d'un champ électrique constant.  Troisièmement, régler le courant dans l'aimant (bobines de Helmholtz) pour produire une déviation nulle du faisceau d'électrons.  Le vecteur du champ magnétique, B, doit être perpendiculaire à la fois à v (la vitesse des électrons) et à E (le champ électrique).\\
\par En utilisant les mesures des champs électrique et magnétique, ainsi que les mesures géométriques de la trajectoire, le rapport e/m de l'électron peut être calculé.
\subsection{Théorie}
Si l'électron est une particule de charge finie e et de masse m, il obéit aux lois du mouvement. \\
\par Dans le tube TEL 525, les électrons sont accélérés à travers le canon à électrons en traversant une différence de potentiel, V. Les électrons sortent du canon à électrons avec une énergie cinétique accrue résultant de la conservation de l'énergie :\\
$$\frac{1}{2}mv^2 = qV$$\\
où :\\
\begin{enumerate}
    \item v est la vitesse dans la direction x, 
    \item q est la charge de l'électron, et 
    \item V est la différence de potentiel. 
\end{enumerate}
Si la vitesse des électrons sortant du canon est connue, le rapport e/m peut être calculé en utilisant l'équation ci-dessus (1) où $q = e$ .\\
\par Le champ électrique dans la direction -y va dévier les électrons vers le haut de sorte qu'une déviation dans la direction y en fonction de x (ou L dans ce cas) peut être observée.  La déviation dans la direction y peut être calculée à l'aide de la deuxième loi de Newton, eE = ma.  En résolvant l'accélération dans la direction +y, nous trouvons que\\
$$a_y = \frac{e}{m}E_y.$$ 
La déviation dans la direction y peut être calculée à partir de\\
$$y = v_{oy}t + \frac{1}{2}at^2,$$
le déplacement dû à une accélération constante.\\    En substituant l'accélération $a_y$ et $v_{oy}$ dans cette équation, on obtient \\
$$y = \frac{eEL^2}{2mv_x^2}$$
où\\
\begin{enumerate}
    \item y est la déviation du faisceau d'électrons,
    \item e est la charge électrique (1,602x10-19 coulombs),
    \item E est le champ électrique (volts/mètre), 
    \item L est la distance horizontale parcourue à travers le champ électrique vertical,
    \item vx est la vitesse des électrons sortant du canon à électrons, et
    \item e/m est le rapport charge/masse.
\end{enumerate}
En dehors du rapport charge-masse (e/m), la vitesse de l'électron, $v_x$, est la seule inconnue de l'équation de déviation (2).  Pour mesurer $v_x$, nous croisons \textit{le champ électrique vertical avec un champ magnétique horizontal} afin de produire un faisceau à déflexion nulle.  La déviation due au champ électrique vertical est annulée par la déviation due au champ magnétique horizontal.  Puisque la somme des forces externes est nulle en cas de "déflexion nulle", la force électrique doit être égale à la force magnétique (force de Lorentz).\\
$$e\overrightarrow{E} = e\overrightarrow{v} \times \overrightarrow{B}$$
A partir de cette équation, nous trouvons que\\
$$v_x = \frac{E}{B}$$
où :\\
\begin{enumerate}
    \item E est le champ électrique de déviation (volts/mètre), et 
    \item B est le champ magnétique (tesla) nécessaire pour annuler la déviation. 
\end{enumerate}
En substituant l'équation (4) à l'équation (2), on trouve que la déviation est : y=emB2L22E .                  (5) En utilisant l'équation (5), nous pouvons maintenant déterminer le rapport e/m. 

Traduit avec www.DeepL.com/Translator (version gratuite)

\end{document}

 