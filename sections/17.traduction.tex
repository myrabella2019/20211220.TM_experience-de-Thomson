\documentclass[../main.tex]{subfiles}
\graphicspath{{\subfix{../images/}}}
\begin{document}
\section{Traduction}

\href{https://physicsx.erau.edu/HelmholtzCoils/Lab_MP_1.pdf}{e/m ratio of the electron}
    
\subsection{Objectif pédagogique}
Étudier le rapport charge-masse (e/m) des rayons cathodiques (électrons) dans un tube cathodique (CRT). 
\subsection{Objectif expérimental}
Mesurer la déviation d'un faisceau d'électrons dans un champ électrique et magnétique croisé et déterminer le rapport charge/masse.  
\subsection{Appareil}
Un tube cathodique Tel-Atomic contenant :
\begin{enumerate}
    \item un "canon à électrons", 
    \item une grille pour mesurer la déviation du faisceau d'électrons, et 
    \item une grille de champ électrique pour dévier les électrons. 
    \item Une paire de bobines de Helmholtz pour produire un champ magnétique homogène.   
\end{enumerate}
Des voltmètres et des ampèremètres pour : \\
\begin{enumerate}
    \item surveiller les potentiels sur le canon à électrons et la grille de déviation, et
    \item surveiller le courant dans le champ magnétique (bobines de Helmholtz).
\end{enumerate}
\subsection{Méthode}
Premièrement, accélérer les électrons à travers le "canon à électrons".  Deuxièmement, les dévier à l'aide d'un champ électrique constant.  Troisièmement, régler le courant dans l'aimant (bobines de Helmholtz) pour produire une déviation nulle du faisceau d'électrons.  Le vecteur du champ magnétique, B, doit être perpendiculaire à la fois à v (la vitesse des électrons) et à E (le champ électrique).\\
\par En utilisant les mesures des champs électrique et magnétique, ainsi que les mesures géométriques de la trajectoire, le rapport e/m de l'électron peut être calculé.
\subsection{Théorie}
Si l'électron est une particule de charge finie e et de masse m, il obéit aux lois du mouvement. \\
\par Dans le tube TEL 525, les électrons sont accélérés à travers le canon à électrons en traversant une différence de potentiel, V. Les électrons sortent du canon à électrons avec une énergie cinétique accrue résultant de la conservation de l'énergie :\\
$$\frac{1}{2}mv^2 = qV$$\\
où :\\
\begin{enumerate}
    \item v est la vitesse dans la direction x, 
    \item q est la charge de l'électron, et 
    \item V est la différence de potentiel. 
\end{enumerate}
Si la vitesse des électrons sortant du canon est connue, le rapport e/m peut être calculé en utilisant l'équation ci-dessus (1) où $q = e$ .\\
\par Le champ électrique dans la direction -y va dévier les électrons vers le haut de sorte qu'une déviation dans la direction y en fonction de x (ou L dans ce cas) peut être observée.  La déviation dans la direction y peut être calculée à l'aide de la deuxième loi de Newton, eE = ma.  En résolvant l'accélération dans la direction +y, nous trouvons que\\
$$a_y = \frac{e}{m}E_y.$$ 
La déviation dans la direction y peut être calculée à partir de\\
$$y = v_{oy}t + \frac{1}{2}at^2,$$
le déplacement dû à une accélération constante.\\    En substituant l'accélération $a_y$ et $v_{oy}$ dans cette équation, on obtient \\
$$y = \frac{eEL^2}{2mv_x^2}$$
où\\
\begin{enumerate}
    \item y est la déviation du faisceau d'électrons,
    \item e est la charge électrique (1,602x10-19 coulombs),
    \item E est le champ électrique (volts/mètre), 
    \item L est la distance horizontale parcourue à travers le champ électrique vertical,
    \item vx est la vitesse des électrons sortant du canon à électrons, et
    \item e/m est le rapport charge/masse.
\end{enumerate}
En dehors du rapport charge-masse (e/m), la vitesse de l'électron, $v_x$, est la seule inconnue de l'équation de déviation (2).  Pour mesurer $v_x$, nous croisons \textit{le champ électrique vertical avec un champ magnétique horizontal} afin de produire un faisceau à déflexion nulle.  La déviation due au champ électrique vertical est annulée par la déviation due au champ magnétique horizontal.  Puisque la somme des forces externes est nulle en cas de "déflexion nulle", la force électrique doit être égale à la force magnétique (force de Lorentz).\\
$$e\overrightarrow{E} = e\overrightarrow{v} \times \overrightarrow{B}$$
A partir de cette équation, nous trouvons que\\
$$v_x = \frac{E}{B}$$
où :\\
\begin{enumerate}
    \item E est le champ électrique de déviation (volts/mètre), et 
    \item B est le champ magnétique (tesla) nécessaire pour annuler la déviation. 
\end{enumerate}
En substituant l'équation (4) à l'équation (2), on trouve que la déviation est :\\
$$y= \frac{e \cdot B^2 L^2}{m \cdot 2E}.$$
En utilisant l'équation (5), nous pouvons maintenant déterminer le rapport e/m. 

\subsection{Procédure}
\begin{enumerate}
    \item Allumez les ampèremètres. 
    \item Tournez les boutons de réglage grossier et fin des alimentations haute tension (5000V et 2000V) complètement dans le sens inverse des aiguilles d'une montre pour vous assurer que la tension est au minimum avant de les mettre en marche.  Les boutons de réglage "fin" peuvent être cassés sur certaines des alimentations.  Vous n'avez pas besoin de les utiliser. 
    \item Mettez en marche les deux alimentations haute tension.  \textbf{Remarque : l'une des alimentations haute tension fournit une tension à la fois au canon à électrons (HV) et au filament (6VAC), (c'est-à-dire à la lampe à incandescence qui fournit la source d'électrons). }
    \item Réglez le potentiel du "canon à électrons" à l'aide du bouton grossier jusqu'à ce que la tension se situe entre 2000 et 3000 volts. 
    \item Ajustez ensuite le potentiel de la "grille de déviation" à l'aide du bouton grossier jusqu'à ce que la déviation soit proche du maximum (environ 2,5 cm dans la direction des y pour 10 cm dans la direction des x).  
    \item Enregistrez la tension du canon à électrons :... volts.  
    \item Enregistrez la tension de la grille : ....volts (V) et son incertitude .... volts (δV).  
    \item Calculez le champ électrique (E) :  $E=V/d$ .... V/m et son incertitude .... V/m.  \textit{(Voir la section sur l'analyse des erreurs)}
    \item Enregistrez la déviation (\textit{y}) : .... m et son incertitude ....m.\\ Enregistrez la longueur (\textit{L}) :   .... m et son incertitude .... m. 
    \item Ajustez le courant dans les bobines de Helmholtz jusqu'à ce que la déviation soit nulle.  Relever le courant sur l'ampèremètre.  
    \item Enregistrez le courant dans l'aimant : ... A et son incertitude ... A. \\
Afin de calculer le champ magnétique, utilisez l'équation suivante pour le champ B dans une bobine de Helmholtz :\\ $$B = \frac{8n \cdot I \mu_o}{5 \sqrt{5} \cdot a} [tesla]$$
où\\
\begin{enumerate}
    \item n = nombre de tours = 320 tours 
    \item I = le courant (ampères) 
    \item a = rayon moyen (0,068 m)
    \item µo= 4πx10-7 webers/(ampère x mètre) 
    \item B = champ magnétique (tesla) \\
    Remarque : 1 tesla = 1 weber / mètre$^2$. 
\end{enumerate}
\item Calculez le champ magnétique (B). .... tesla.  Calculez l'incertitude du champ magnétique ..... tesla.  
\item Calculer la vitesse des électrons en utilisant $v_x = E/B$.... m/s.
\item Calculez le rapport e/m. ... C/kg.  
\item Calculer l'incertitude sur le rapport e/m : .... C/kg. 
\end{enumerate}
\subsubsection{Question #1}
Comment le rapport e/m calculé à partir de cette expérience se compare-t-il à la valeur connue de e/m ?  \subsubsection{Question n°2}
Comment l'intensité du champ magnétique dû aux bobines de Helmholtz se compare-t-elle au champ magnétique local dû à la terre ? 
\subsubsection{Question n°3} Quelle est votre réponse finale pour le rapport charge-masse ?  Vous devez l'exprimer sous la forme suivante : 
$$e/m = (e/m)_{calculated} +- \delta(e/m)$$
Voir la section Analyse des erreurs pour plus de détails. 
\subsubsection{Question n°4}
Que recommanderiez-vous pour améliorer la mesure du rapport e/m que vous venez d'effectuer ?
\subsection{Analyse des erreurs}
Après avoir enregistré vos mesures et leurs incertitudes respectives, il est maintenant possible de calculer l'incertitude de la mesure finale.  À titre d'exemple, calculons l'erreur relative du champ électrique entre les plaques de déviation.  Le champ électrique est calculé à partir de l'équation :\\
$$E=\frac{V}{d}$$ 
où V est le potentiel (volts) et \textit{d} est la séparation entre les plaques (mètres).  L'incertitude relative du champ électrique peut être calculée à l'aide de la formule suivante :\\
$$\frac{\delta E}{E} = \sqrt{(\frac{\delta V}{V})^2 + (\frac{\delta d}{d})^2}$$
où les incertitudes relatives du potentiel et de la séparation sont ajoutées en quadrature.\\
Lorsque vous donnez votre réponse pour e/m (le rapport charge-masse), vous devez indiquer à la fois la valeur moyenne et l'incertitude.  L'équation permettant de calculer la valeur moyenne de e/m est la suivante :\\
$$\frac{e}{m} = \frac{2Ey}{B^2L^2}$$
où B est le champ magnétique (tesla), E est le champ électrique (volts/mètre), y est le déplacement (mètres) et L est la longueur de la grille de déviation (mètres).\\
Pour déterminer l'incertitude de e/m, vous devez d'abord calculer l'incertitude relative de e/m.\\
$$\frac{\delta (e/m)}{e/m} = \sqrt{(\frac{\delta y}{y})^2 + (\frac{\delta E}{E})^2 + (\frac{2 \delta B}{B})^2 +(\frac{2 \delta L}{L})^2 }$$
où les incertitudes relatives des quantités mesurées sont additionnées en quadrature.\\
La réponse finale doit être citée comme suit :\\
$$e/m = (e/m)_{calculée} +- \delta(e/m)$$
 \subsection{Le champ magnétique}
Le champ magnétique pour l'une des deux bobines de Helmholtz peut être écrit à l'aide de la loi de Biot-Savart :\\
$$B_z = nI \times 10^{-7}\int_0^{2\pi}\frac{a(a -rcos\theta) d\theta}{(a^2 + b^2 + r^2 - 2arcos\theta)^{3/2}}$$ où n est le nombre de spires, a est le rayon de la bobine (mètres), b est la distance par rapport au plan de la bobine (mètres) et r est la distance par rapport à l'axe de symétrie (perpendiculaire au plan de la bobine et passant par le centre de la bobine).  Toutes les bobines de Helmholtz sont orientées de telle sorte que la séparation entre les deux bobines est a (le rayon d'une bobine individuelle).  À cette distance, le champ magnétique dans le plan médian entre les deux bobines (c'est-à-dire à b = a/2) est très homogène.  Ceci est illustré dans la figure ci-dessous.  Cependant, à mesure que l'on s'éloigne de l'axe de symétrie (r=0) et que l'on se rapproche du rayon de la bobine de Helmholtz (r → 0,068 mètre), on observe que le champ magnétique diminue rapidement.    B (tesla)r (mètres) Champ magnétique en fonction de r La figure ci-dessus montre le champ magnétique (B en tesla) dans le plan médian entre les deux bobines en fonction de r (la distance à l'axe de symétrie) pour n = 320 tours, i = 170 mA et b = a/2, où a = 0,068 mètres.    Si vous prenez l'équation ci-dessus pour Bz (éq. 8), que vous définissez b = a/2 et r = 0, que vous l'intégrez et que vous la multipliez par deux (parce qu'il y a deux bobines), vous obtiendrez l'équation B=32!n55Ia "10#7 (tesla) comme indiqué ci-dessus (éq. 6). 

Traduit avec www.DeepL.com/Translator (version gratuite)
Traduit avec DeepL (deepl.com)

\end{document}

 