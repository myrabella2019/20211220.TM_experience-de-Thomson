newpage
\section{Rédaction de TP : le squelette}
\subsection{Le but :}
Précise les objectifs très généraux à atteindre pendant la séance de TP : quelques lignes suffisent.
\subsection{La théorie : }
Doit contenir un développement des bases théoriques sur lesquelles on s'appuie pour réaliser l'expérience. Il s'agit d'un résumé pertinent et personnel.
\subsection{Les hypothèses de départ : }
Décrit, point par point quelles sont les conditions dans lesquelles on entend mener l'expérience, dans quelles conditions la loi prédéfinie est valable et par-dessus tout quels sont les phénomènes que l'on négligera.
\subsection{La marche à suivre : }
Décrit, comment, au moyen du montage, le but peut être atteint ; elle justifie le montage et la méthode utilisée ; dans cette marche à suivre, on indiquera clairement toutes les grandeurs physiques que l'on va mesurer ainsi que toutes les valeurs provenant de tables numériques.
\subsection{Les mesures : }
Sont des valeurs numériques que l'on obtient au moyen d'instruments ; on rassemblera si possible les mesures dans un tableau ; on estimera les incertitudes des mesures.
\subsection{Les résultats : }
sont obtenus à partir des mesures, soit par calcul, soit par méthode graphique.\\
En cas de calcul, on présentera clairement un exemple développé du calcul d'incertitude.\\
Lorsque l'on tracera un graphique, on indiquera :
\begin{itemize}
    \item un titre
    \item les grandeurs reportées sur les axes avec leurs unités
    \item des échelles claires et précises sur chaque axe
    \item les barres et/ou rectangles d'incertitudes
    \item la courbe reliant les valeurs si elle existe
    \item la courbe théorique si on la connaît.
\end{itemize}
\subsection{La conclusion : }
Consiste à discuter si le but est atteint ou non ;\\
on discutera notamment les points suivants :
\begin{itemize}
    \item les valeurs obtenues sont-elles en accord avec celles des tables et dans la négative, d'où provient la différence ? En accord signifie compatible avec les incertitudes, d'où  l'importance de ces dernières !
    \item existe-t-il un modèle physique (une relation mathématique) reliant les grandeurs physiques étudiées ? Laquelle ?
    \item la méthode choisie est-elle bien adaptée ?
    \item le montage permet-il  d'atteindre une bonne précision ?
    \item les hypothèses de départ s'avèrent-elles légitimes ?
    \item ...
\end{itemize}