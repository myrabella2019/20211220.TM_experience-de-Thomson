\documentclass[../main.tex]{subfiles}
\graphicspath{{\subfix{../images/}}}
\begin{document}
\bibliographystyle{plain}
\bibliography{livre_phy.bib}
\section{16.10 - L'oscilloscope - Livre de physique p.421 . }
    
À l'exception de quelques appareils de mesure simples, aucun instrument scientifique n'est probablement d'un usage aussi répandu que \textit{l'oscilloscope}. Les circuits électriques contenus dans un oscilloscope sont très compliqués, et les modèles sophistiqués ont tellement d'interrupteurs et de boutons de contrôle qu'il faut certain temps pour se familiariser avec son maniement. Cependant, les principes de base de la composante principale, le \textit{tube cathodique}, peuvent être compris entièrement grâce aux concepts développés dans ce chapitre.\\
\par Dans la première partie du tube, qui est le canon à électrons, une différence de potentiel constance accélère des électrons émis par un filament chaud (figure 16.26). S'il n'y a pas d'autre forces sur l'électron, ils se déplacent en ligne droite et frappent le centre de l'écran fluorescent, produisant un point brillant appelé spot, d'après le terme anglais. On peut produire une déflexion horizontale de ce spot, au moyen d'un champ électrique horizontale engendré par les lames parallèles marquées 1 sur la figure 16.26. Ces lames sont distantes de \textit{l}. Si elles sont à une différence de potentiel $V_1$, il existe un champ électrique horizontal uniforme $E_1 = V_1/l$ entre elles. Ce champ accélère les électrons, qui frappent alors l'écran à une distance horizontale du centre proportionnelle à $V_1$. Si on accroît graduellement $V_1$, le spot se déplace progressivement ou \textit{balaie} l'écran ; quand $V_1$ reprend sa valeur originale, le spot retourne à son point de départ. Si ce balayage est répété à une fréquence suffisamment élevée, la persistance de l'image sur l'écran et dans l'oeil cache le mouvement et on perçoit un segment de ligne droite (figure 16.27\textit{b})\\
\subsubsection*{Figure 16.26}
Un tube cathodique. Les électrons sont accélérés de la cathode négative vers l'anode accélératrice positive et passent alors entre deux paires de plaques de déflexion. Quand ils frappent l'écran fluorescent, il y a émission de lumière.\\
\par De la même façon, une différence de potentiel $V_2$ appliquée aux plaques marquées 2 sur la figure 16.26 produira une déflexion verticale. Par exemple, un $V_2$ constant déplacera la ligne comme indiqué sur la figure 16.27c. Habituellement la déflexion verticale est produite par une différence de potentiel inconnue $V_2$, qui oscille à une fréquence déterminée \textit{f}. On ajuste la fréquence de balayage jusqu'à ce qu'elle soit égale à \textit{f} ( ou \textit{f} divisée par un nombre entier), de sorte que chaque fois que le signal vertical $V_2$ se répète, le balayage horizontal est au même point de son cycle. Donc le faisceau frappe de façon répétée les mêmes points sur l'écran et on distingue alors une forme stable (figure 16.27d,e). De cette façon, on peut aisément opérer des mesures précises tant de la fréquence que de l'amplitude de la différence de potentiel inconnue.\\
\par Puisqu'il est possible de convertir presque toutes les sortes d'information en différences de potentiel électrique, on ut lise les oscilloscopes dans pour ainsi dire tous les types de laboratoire. Ils extrêmement précieux dans les études qualitatives et quantitatives, non seulement de variables électriques, mais aussi de grandeurs mécaniques acoustiques et autre. Les microphones et les caméras de télévision sont des exemples de dispositifs qui convertissent d'autres formes d'énergies en potentiels électriques variables. En général, les dipositifs qui convertisesnt nue forme d'énergie en une autre sont appelés des \textit{capteurs}.\\
\subsubsection{Figure 16.27}
\begin{enumerate}[(a)]
    \item S'il n'y a pas de champs déflecteurs, on voit une tache brillante au centre de l'écran de l'oscilloscope.
    \item En changeant progressivement la différence de potentiel horizontale, on force la tache du faisceau à se déplacer progressivement sur l'écran. Si ce balayage est répété assez vite, on perçoit une ligne continue
    \item Un champ constant vertical déplace la ligne.
    \item Si le champ vertical se renverse après chaque demi-balayage, on voit la figure représentée ici.
    \item On peut mettre en évidence des variations temporelles du champ vertical si elles se répètent à la fréquence de balayage.
\end{enumerate}
\end{document}