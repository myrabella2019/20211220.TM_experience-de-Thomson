\documentclass[../main.tex]{subfiles}
\graphicspath{{\subfix{../images/}}}
\begin{document}
\section{Mémento - Dates de rdv et de rendus - notes écrites sur la feuille}
\subsection{Janvier / début février 2022}
Tout les rdvs sont les jeudis à midi\\
rdv : 20 janvier \\
rdv : 27 janvier \qquad temps entre les 2, c'est le temps qu'elle met pour lire \\
\subsection{Février / Mars 2022}
faire attenion à la fiabilité - consulter des livres\\ 
avoir la marche à suivre et la théorie de l'expérience\\
rdv le 17 mars 2022\\
rdv le 24 mars 2022\\
\subsection{Fin Mars / Avril 2022}
rdv le 5 mai 2022\\
rdv le 12 mai 2022\\
\subsection{Mai 2022}
Vérification de la réponse à la problématique(on va dans la bonne direction...)\\
très indulgente sur l'orthographe, au début elle corrige les fautes d'orthographe mais ensuite elle nous laisse faire cette tâche car elle estime qu'on en est assez capable.\\
rdv le 26 mai 2022\\
\subsection{Après les vances d'été en septembre}
rdv le 8 septembre 2022\\
\subsection{le 10 octobre 2022, Lundi matin à 10h}
remise au secrétariat du texte final imprimé en 2 exemplaires + version informatique (pdf envoyé par e-mail au secrétariat)
\subsection{Jeudi 22 ou vendredi 23 décembre 2022}
(20 minutes de présentation orale et 15 minutes de questions)\\
il faut apporter qqch de nouveau par rapport à l'écrit \\
\subsection{Remarques générales}
\begin{enumerate}
    \item tenir un carnet de bord qui contient le travail fait au jour le jour et qu'il faut présenter à chaque entretien
    \item une trace écrite des entretiens (conservée par le tuteur), cette dernière remarque n'a jamais vraiment été réalisée. Si elle voit qu'on est très désorganisé dans notre travail, elle le fera, mais en réalité elle ne la pratiquement jamais fait
    \item dès le début, garder la trace des sites consultés, sur latex, ne pas oublier d'indiquer la date de consultation, car les sites changent toujours \textbf{avoir un glossaire, surtout pour moi qui fait sur une expérience scientique utilisant des mots très techniques, tout à la fin du TM il faut les définir, en plus que je vais organiser un TP pour les premières, ils ne doivent pas être perdus.}
    \item lire attentivement les documents distribués.
    \item consulter d'anciens TM, le mien a déjà été réalisé avec Rebstein, un groupe avait fait sur le dentifrice d'éléphant, ils sont disponible en bibliothèque
    \item Faire très attention au plagiat, un document nous a été distribué\\
    
\end{enumerate}

\end{document}