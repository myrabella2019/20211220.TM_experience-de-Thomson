\documentclass[../main.tex]{subfiles}
\graphicspath{{\subfix{../images/}}}
\begin{document}
\bibliographystyle{plain}
\bibliography{livre_phy.bib}
\section{p.493 - texte NON reformulée}
\subsection*{19.9 - La mesure du rapport de la charge à la masse}
à11h commencé
Dans ce paragraphe et dans les deux suivants, nous allons étudier quelques utilisations de la force magnétique sur une charge en mouvement. Nous discuterons ici des arrangements de champs électrique et magnétique que l'on peut employer pour mesurer le rapport de la charge à la masse d'une particule chargée.\\
\par La figure 19.31 montre une particule de masse \textit{m} et de charge \textit{q} accélérée à partir du repos par une différence de potentiel \textit{V} connue. Puisque la somme des énergies cinétique et potentielle doit rester constante, la vitesse finale \textbf{v} obéit à (1/2)$mv^2 = qV$ et le rapport de la charge à la masse vaut : \\
$$\frac{q}{m} = \frac{v^2}{2V}$$
Ainsi donc, on peut trouver le rapport $q/m$ de la charge à la masse pour la particule considérée, si l'on est capable d'en mesurer la vitesse. On peut effectuer cette mesure au moyen de champs électrique et magnétique \textit{croisés}, c'est-à-dire perpendiculaires l'un à l'autre, comme le montre la figure 19.31. La force magnétique agissant sur la particule est $q\textbf{v} \times \textbf{B}$ tandis que la force électrique vaut $q\textbf{E}$. Puisque $\textbf{v \times B}$ est orienté vers le haut, c'est-à-dire dans le sens opposé à \textbf{E}, ces deux forces se contrebalancent exactement quand $qvB = qE$, ou \\
$$v = \frac{E}{B}$$ \\
\subsubsection*{Figure 19.31}
La particule chargée est accélérée du repos à la vitesse \textbf{v} par une différence de potentiel \textit{V}. Le champ magnétique \textbf{B} est orienté dans le sens entrant dans le plan du dessin. Le vecteur $\textbf{v \times B}$ est dans le sens opposé à \textbf{E}, de sorte que la force résultante $\textbf{F} = q\textbf{E} + q\textbf{v} \times \textbf{B}$ est nulle pour une valeur déterminée de E, B.\\
\par On peut donc déterminer la vitesse en ajoutant les champs croisés pour ramener le spot sur l'écran fluorescent à la position qu'il occupait en l'absence de champ.\\
J.J. Thomson (1856-1940) a utilisé un dispositif de ce genre pour mesurer le rapport de la charge à la masse des rayons cathodiques négatifs d'un tube cathodique. Il conclut que ces rayons cathodiques étaient des particules chargées négativement (électrons). Il trouva également que leur vitesse, dans son appareil, était environ un dixième de la vitesse de la lumière, ce qui est bien supérieur aux vitesses observées auparavant.\\
\par Une autre propriété de cet appareillage vaut d'être signalée. Les champs \textbf{E} et \textbf{B} permettent à des particules chargées de les traverser sans déflexion si la condition $v  = E/B$ est satisfaite, quelles que soient la charge et la masse de ces particules. Si un faisceau de plusieurs sortes de particules chargées, douées de vitesses différentes, passe au travers d'une telle région de champs croisés, les particules qui en sortent sans avoir subi de déflexion ont toutes la même vitesse. Ainsi donc, des champs électrique et magnétique croisés se comportent en \textit{sélecteur de vitesse}. Une utilisation d'un tel dispositif est décrite au paragraphe suivant.
\subsection*{19.10 - Spectromètres de masse}
à 12h41
Le \textit{spectromètre de masse} a été initialement développé dans des buts de recherche en physique nucléaire. Aujourd'hui, les spectromètres de masse sont couramment (12h58) utilisés dans de nombreux types de laboratoires pour mesurer et identifier de minimes quantités de substances diverses.\\
\par On peut comprendre le principe du spectromètrede masse en considérant une particule de charge positive \textit{q} et de masse \textit{m} qui se déplace perpendiculairement  à un champ magnétique uniforme \textbf{B} (figure 19.32). La force magnétique $\textbf{F} = q\textbf{v} \times \textbf{B}$ est perpendiculaire à la vitesse \textbf{v}, de sorte qu'elle en change la direction, mais pas la grandeur. Le module de la force, \textit{qvB}, reste inchangé puisque \textbf{v} et \textbf{B} sont constants en grandeur et perpendiculaires l'un à l'autre.\\
\subsubsection*{Figure 19.32}
Une particule chargée se déplace avec une vitesse \textbf{v}, perpendiculairement à un champ magnétique uniforme \textbf{B} orienté dans le sens entrant dans le plan du dessin. Comme la force et l'accélération sont constamment perpendiculaires à v et restent inchangées en module, la particule est en mouvement circulaire uniforme.\\
\par Puisque l'accélération $\textbf{a} = \textbf{F}/m$ est constante en grandeur et reste toujours perpendiculaire à \textbf{v}, la particule décrit une orbite circulaire à vitesse constante. L'accélération centripète $a_r = v^2/R$ est assurée par la force magnétique. Aussi le rayon \textit{R} de l'orbite doit-il satisfaire à $qvB = mv^2/R$, ce qui donne \\
$$R = \frac{mv}{qB}$$ \\
L'exemple suivant illustre l'emploi de cette formule.
\subsubsection{Exemple 19.8}
Quel champ magnétique faut-il pour qu'un ion $O_2^+$ se déplace sur une orbite circulaire de rayon 2 m, sa vitesse étant de $10^6 [ms^{-1}]$ ? (La masse de l'ion $O_2^+$ est approximativement 32 uma, où 1 uma = $1.66 \times 10^{-27}$ kg.)\\
\textbf{RÉPONSE} \quad La charge de l'ion est $e = 1.60 \times 10^{-19}$ C. En utilisant $R = mr/qB$, on obtient que le champ magnétique vaut \\
$$B = \frac{mv}{qR}$$ \\
$= \frac{(32 \times 1.66 \times 10^{-27}[kg])(10^6 [ms^{-1}])}{(1.60 \times 10^{-19}[C])(2 m)}$\\
$= 0.167 T$\\
\subsubsection*{Figure 19.33}
Un spectromètre de masse. Des champs croisés, $E_1$ et $B_1$ sélectionnent les particules qui ont une vitesse donnée. Le rayon de la trajectoire circulaire dans le champ \textbf{B} détermine alors $m/q$. \\
\par La figure 19.33 montre les principaux composants d'un spectromètre de masse. Dans la source d'ions, on ionise les molécules en les bombardant avec des électrons ; les ions sont ensuite extraits de la source par un champ électrique . Dans le sélecteur de vitesse, des champs électrique et magnétique croisés  permettent aux seuls ions de vitesse $v = E_1/B_1$ de ne pas subir de déflexion et donc d'atteindre la fente d'entrée  dans la région de champ uniforme \textbf{B}. Dans cette région, les ions se déplacent sur des trajectoires circulaires de rayon $R = mv/qB$, jusqu'à ce qu'ils atteignent la plaque à des distances augmentant proportionnellement  à leur masse, puisque le rayon de courbure croît avec cette masse. Ceci permet de séparer les ions de même charge mais de masse différente. \\
\par Historiquement, le spectromètre de masse rendit possible l'étude systématique des isotopes. Les isotopes sont des formes distinctes d'un même élément, possédant presque exactement les mêmes propriétés chimiques, mais dont les noyaux ont des nombres de neutrons différents, ce qui entraîne des masses atomiques différentes. Plus tard, le spectromètre de masse fut également utilisé dans le cas de l'uranium, pour séparer l'isotope 235 fissible (à13h56) (à15h05) du $^{238}U$, plus abondant, ceci  dans le cadre du développement d'armes nucléaires pendant la Deuxième Guerre mondiale.
\\
\par La capacité du spectromètre de masse à différencier les isotopes d'un corps en fait un outil de recherche inestimable dans les études qui impliquent des isotopes stables plutôt que des isotopes \textit{radioactifs}. Ces derniers se transforment spontanément en d'autres espèces nucléaires, en émettant un rayonnement ionisant. Si on administre une substance radioactive  à une plante ou à un animal, on peut suivre son trajet dans l'organisme en détectant l'ionisation produite par le rayonnement. Ces \textit{traceurs radioactifs} sont universellement utilisés en recherche biologique et dans le diagnostic médical. Cependant, des éléments importants biologiquement, comme l'azote et l'oxygène, ne possèdent pas d'isotopes radioactifs appropriés, alors qu'ils ont évidemment des isotopes stables. Par exemple, l'oxygène normale comporte 99.756 \% d'oxygène 16, qui est un noyau contenant 8 protons et 8 neutrons ; 0.039 \% d'oxygène 17, qui a un neutron de plus ; et 0.205 \% d'oxygène 18, qui a encore un neutron en plus. Si on administre à un organisme vivant des substances contenant de l'oxygène enrichi en un ou plusieurs de ces isotopes rares, on peut prélever, à différents moments, des échantillons à cet organisme et les analyser avec un specrtomètre de masse. La présence des isotopes rares, de l'oxygène signale l'arrivée de la substance administrée ou de ses dérivés métaboliques. Ainsi donc, le spectromètre de masse rend possible l'utilisation d'isotopes stables comme traceurs. On l'utilise aussi pour identifier les rapports d'abondance d'isotopes stables dans des échantillons géologiques, dans le but d'aider à la détermination de leur source ou de leur âge.\\
\par Les masses des isotopes d'un élément déterminé ont des différences relatives appréciables pour les éléments les plus légers ; dans ce cas, des variations significatives, bien que faibles, se manifestent parfois dans les vitesses de réactions chimiques, d'évaporation, etc. Il en résulte que des variations mesurables apparaissent dans les rapports des isotopes, entre autres, de l'hydrogène, du carbone et de l'oxygène présents dans les minéraux, les grandes masses d'eau et les organismes vivants. De nombreuses recherches contemporaines en biologie, en géologie, en océanographie et en archéologie sont rendues possibles par l'emploi des spcetromètres de masse pour étudier ces variations minimes de rapports isotopiques. Par exemple, des chercheuses sont remontés à la source du carbone utilisé par le plancton dans un lac et ont pu, par cette méthode, trier les fragments de plusieurs colonnes grecques portant des inscriptions.\\
\par On emploie parfois des spectromètres de masse dans des cas où on ne s'intéresse pas à la composition isotopique. On les a utilisés pour analyser les gaz exhalés par des patients présentant des affections diverses et pour étudier la composition du gaz au voisinage de plantes durant la photosynthèse. On les utilise aussi dans l'industrie pétrolière pour distinguer des composés complexes qui ont des compositions chimiques identiques mais des configurations moléculaires différentes et un spectromètre de masse fut l'un des instruments emportés par les sondes spatiales qui se posèrent sur Mars et sur Vénus.
\subsection*{Cyclotrons}
Le \textit{cyclotron}, inventé par Ernest O. Lawrence (1901-1958) en 1930, fut la première machine réalisée pour accélérer des particules chargées jusqu'à des vitesses élevées, en les faisant passer de façon répétée à travers la même région d'accélération. Son fonctionnement  repose sur le fait remarquable que la période, c'est-à-dire le temps requis par une particule chargée pour effectuer un tour complet sur une orbite circulaire dans un champ magnétique uniforme \textbf{B}, est indépendante de la vitesse \textit{v} de la particule.\\
\par Au paragraphe précédent, nous avons montré que, pour une particule de charge \textit{q} et de masse \textit{m}, le rayon \textit{R} de l'orbite est $R = mv/qB$. La période \textit{T} obéit à $vT = 2\pi R$, ce qui donne \\
$$T = \frac{2 \pi R}{v} = \frac{2 \pi m}{qB}$$
\\
Donc, accroître la vitesse augmente le rayon de l'orbite, mais n'a pas d'effet sur la période \textit{T} ou sur la fréquence orbitale $f = 1/T$.\\
\par Un cyclotron consiste en deux demi-cylindres (appelés "dees") de métal, creux et dans lesquels on a fait le vide. Ils sont placés dans un champ magnétique uniforme, perpendiculaire au plan de leur base (figure 19.34). Des protons ou d'autres ions positifs sont injectés près du centre. Un générateur électrique inverse la différence de potentiel entre les demi-cylindres à la fréquence du mouvement orbital des ions, de sorte qu'ils sont accélérés à chaque passage dans l'intervalle entre ces demi-cylindres. On augmente ainsi leur vitesse et, par conséquent, le rayon de leur orbite $R = mv/qB$, sans cependant modifier la période du mouvement.\\
\par Le fonctionnement du cyclotron repose sur le fait que la période ne dépend pas de la vitesse. Cependant, on trouve que la masse d'inertie d'une particule augmente rapidement quand sa vitesse s'approche de la vitesse de la lumière. Ce fait a été prédit pour la première fois en 1905, par Einstein, dans le cadre de sa théorie de la relativité restreinte : cette prédiction fut confirmée par la suite. Cet accroissement de masse entraîne une augmentation de la période et limite la vitesse et l'énergie cinétique maximum que l'on puisse atteindre avec un cyclotron. (10h30) On peut obtenir des énergies plus élevées au moyen d'accélérateurs plus complexes, dans lesquels on fait varier progressivement le champ magnétique ou la fréquence du générateur électrique, au fur et à mesure que les ions sont accélérés.\\
\par Développés initialement pour la recherche en physique nucléaire, les cyclotrons ont été en grande partie remplacés dans ce domaine par des machines plus modernes. Cependant, aujourd'hui, on les utilise quelquefois dans les hôpitaux pour induire, par bombardements de cibles diverses, des réactions nucléaires qui produisent des substances radioactives pour applications médicales. La majorité des produits radioactifs sont en réalité fournis aux hôpitaux par les centres disposant de réacteurs nucléaires. Cependant, certains radio-isotopes ont une vie si courte que l'on ne peut pas les transporter à grande distance. D'autres ne peuvent être produits qu'au moyen de cyclotrons.\\
\subsubsection*{Figure 19.34}
Le principe du cyclotron. Un champ magnétique perpendiculaire au plan des demi-cylindres maintient les ions sur des orbites circulaires. Le rayon de l'orbite augmente au fur et à mesure que les ions sont accélérés par la différence de potentiel entre les demi-cylindre. Cette différence de potentiel est alternative, à la même fréquence que le mouvement orbital des ions.
\end{document}