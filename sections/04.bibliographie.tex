\documentclass[../main.tex]{subfiles}
\graphicspath{{\subfix{../images/}}}
\begin{document}
\section{Bibliographie}
\begin{enumerate}
    \item KANE Joseph, STERNHEIM Morton, Physique, édition Dunod, Licence - Paes
    \item Physique 1, généralités chaleur, édition Lep
    \item BENSON, Physique, 2. Électricité et magnétisme
    \item (LAFRANCE René, Physique 2, électricité et magnétisme, de boeck)
    \item (LAFRANCE René, Physique 3, Ondes, optique et physique moderne, de boeck)
    \item REBSTEIN Marie, livre de chimie pour la maturité et le bac
\end{enumerate}


\\
Vidéos de van biezen : \\
Coulomb's law (1) : https://www.youtube.com/watch?v=-jxX7Vt2wrA&list=PLX2gX-ftPVXUcMGbk1A7UbNtgadPsK5BD&index=1 \\
Coulomb's law (2) : https://www.youtube.com/watch?v=6XR8eHzpwyc&list=PLX2gX-ftPVXUcMGbk1A7UbNtgadPsK5BD&index=2\\
%Coulomb's law (3) : https://www.youtube.com/watch?v=_ZuroDbIi8A&list=PLX2gX-ftPVXUcMGbk1A7UbNtgadPsK5BD&index=3\\
Coulomb's law (4) : https://www.youtube.com/watch?v=lxpPksaNEMU&list=PLX2gX-ftPVXUcMGbk1A7UbNtgadPsK5BD&index=4\\
Coulomb's law example 1(5) : %https://www.youtube.com/watch?v=8_BG3uzh8kg&list=PLX2gX-ftPVXUcMGbk1A7UbNtgadPsK5BD&index=5 \\
Coulomb's law example 2A (6) : https://www.youtube.com/watch?v=d4JcWusjYVM&list=PLX2gX-ftPVXUcMGbk1A7UbNtgadPsK5BD&index=6 \\
Coulomb's law example 3 (7) : https://www.youtube.com/watch?v=q8JVLgyQJZA&list=PLX2gX-ftPVXUcMGbk1A7UbNtgadPsK5BD&index=7 \\
 \href{https://fr.khanacademy.org/science/chemistry/electronic-structure-of-atoms/history-of-atomic-structure/a/discovery-of-the-electron-and-nucleus}{Khan academy}\\

\href{https://chemistrygod.com/cathode-ray-tube-experiments}{ChemistryGod - tube cathodique}\\
Vidéo :
\href{https://www.youtube.com/watch?v=WHoH5m83Ga0}{Académie de Bordeaux, \textit{[Histoire des sciences] La découverte de l'électron}, scientifique, France, CEA Recherche, 15.09.2014}
Pour un ouvrage :
Nom  de  l’auteur,  prénom,  titre  et  sous-titre  (en  italique),  particularité  de l’édition,  lieu  d’édition,  éditeur  (collection),  date  d’édition,  pagination.
Exemple  pour un livre  : Duby  Georges,  Art  et  société  au  Moyen-Age,  Paris,  Seuil  (Points Histoire),  1997,  248p.
Articles  :  par  ordre  alphabétique  du  nom  de  famille  des  auteurs 
Nom  de  l’auteur,  prénom,  titre  de  l’article,  in  titre  du  périodique  (en  italique), numéro,  année  de  parution,  pagination. 
Exemple  pour  un  article  : Verger  Jacques,  Le  Moyen-Age  a  inventé  la  médecine  moderne,  in L’Histoire, no 169, 1993, pp. 20-27 IV.
Sites  internet  :
Les  sites  internet  doivent  être  mis  en  entier  et  merci  d’indiquer  la  date  et l’heure  de  consultation  au  cas  où  le  site  serait  remis  à  jour  ou  aurait  disparu.
V.  Vidéos,
CD-ROM,… Les  films,  reportages,  émissions  de  radios  sont  inscrits  de  la  même  manière que  les  ouvrages  en  remplaçant  le  nom  de  l’éditeur  par  le  nom  de  l’entreprise de  production  du  film  (par  ex.  Gaumont)  ou  par  le  média  qui  a  diffusé l’émission  (par  ex.  RSR1,  Couleur  3). 
\end{document}